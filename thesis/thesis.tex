%%%%%%%%%%%%%%%%%%%%%%%%%%%%%%%%%%%%%%%%%%%%%%%%%%%%%%%%%%%%%%%%%%%%
%% I, the copyright holder of this work, release this work into the
%% public domain. This applies worldwide. In some countries this may
%% not be legally possible; if so: I grant anyone the right to use
%% this work for any purpose, without any conditions, unless such
%% conditions are required by law.
%%%%%%%%%%%%%%%%%%%%%%%%%%%%%%%%%%%%%%%%%%%%%%%%%%%%%%%%%%%%%%%%%%%%

\documentclass[
  digital, %% This option enables the default options for the
           %% digital version of a document. Replace with `printed`
           %% to enable the default options for the printed version
           %% of a document.
  twoside, %% This option enables double-sided typesetting. Use at
           %% least 120 g/m² paper to prevent show-through. Replace
           %% with `oneside` to use one-sided typesetting; use only
           %% if you don’t have access to a double-sided printer,
           %% or if one-sided typesetting is a formal requirement
           %% at your faculty.
  table,   %% This option causes the coloring of tables. Replace
           %% with `notable` to restore plain LaTeX tables.
  lof,     %% This option prints the List of Figures. Replace with
           %% `nolof` to hide the List of Figures.
  lot,     %% This option prints the List of Tables. Replace with
           %% `nolot` to hide the List of Tables.
  %% More options are listed in the user guide at
  %% <http://mirrors.ctan.org/macros/latex/contrib/fithesis/guide/mu/fi.pdf>.
]{fithesis3}
%% The following section sets up the locales used in the thesis.
\usepackage[resetfonts]{cmap} %% We need to load the T2A font encoding
\usepackage[T1,T2A]{fontenc}  %% to use the Cyrillic fonts with Russian texts.
\usepackage[
  main=english, %% By using `czech` or `slovak` as the main locale
                %% instead of `english`, you can typeset the thesis
                %% in either Czech or Slovak, respectively.
]{babel}        %% foreign texts to be typeset as follows:
%%
%%   \begin{otherlanguage}{german}  ... \end{otherlanguage}
%%   \begin{otherlanguage}{russian} ... \end{otherlanguage}
%%   \begin{otherlanguage}{czech}   ... \end{otherlanguage}
%%   \begin{otherlanguage}{slovak}  ... \end{otherlanguage}
%%
%% For non-Latin scripts, it may be necessary to load additional
%% fonts:
\usepackage{paratype}
%%
%% The following section sets up the metadata of the thesis.
\thesissetup{
    date          = \the\year/\the\month/\the\day,
    university    = mu,
    faculty       = fi,
    type          = bc,
    author        = {Tereza Šťastná},
    gender        = f,
    advisor       = {doc. RNDr. Jan Strejček, Ph.D.},
    title         = {Simplification\\ of Acceptance Condition\\ of Emerson-Lei Automata},
    TeXtitle      = {Simplification\\ of Acceptance Condition\\ of Emerson-Lei Automata},
    keywords      = {keyword1, keyword2, ...},
    TeXkeywords   = {keyword1, keyword2, \ldots},
    abstract      = {%
      This is the abstract of my thesis, which can

      span multiple paragraphs.
    },
    thanks        = {%
      These are the acknowledgements for my thesis, which can

      span multiple paragraphs.
    },
    bib           = bibliography.bib,
}
\usepackage{makeidx}      %% The `makeidx` package contains
\makeindex                %% helper commands for index typesetting.
%% These additional packages are used within the document:
\usepackage{paralist} %% Compact list environments
\usepackage{amsmath}  %% Mathematics
\usepackage{amsthm}
\usepackage{amsfonts}
\usepackage{url}      %% Hyperlinks
\usepackage{markdown} %% Lightweight markup
\usepackage{tabularx} %% Tables
\usepackage{tabu}
\usepackage{booktabs}
\usepackage{listings} %% Source code highlighting
\lstset{
  basicstyle      = \ttfamily,
  identifierstyle = \color{black},
  keywordstyle    = \color{blue},
  keywordstyle    = {[2]\color{cyan}},
  keywordstyle    = {[3]\color{olive}},
  stringstyle     = \color{teal},
  commentstyle    = \itshape\color{magenta},
  breaklines      = true,
}
\usepackage{floatrow} %% Putting captions above tables
\floatsetup[table]{capposition=top}
%% The following code fixes the rendering of BibLaTeX ISO 690
%% references in old TeX Live (such as the one at Overleaf).
\thesisload
\makeatletter
\def\thesis@biblatexiso@fix@package{iso-numeric.bbx}
\def\thesis@biblatexiso@fix@end{\relax}
\newif\ifthesis@biblatexiso@fix@
\thesis@biblatexiso@fix@false
\def\thesis@biblatexiso@fix@next#1,{%
  \def\thesis@biblatexiso@fix@current{#1}%
  \ifx\thesis@biblatexiso@fix@current\thesis@biblatexiso@fix@package
    \thesis@biblatexiso@fix@true
  \fi
  \ifx\thesis@biblatexiso@fix@current\thesis@biblatexiso@fix@end
    \expandafter
    \@gobble
  \fi
  \thesis@biblatexiso@fix@next
}
\expandafter\expandafter\expandafter\thesis@biblatexiso@fix@next\@filelist,\relax,
\ifthesis@biblatexiso@fix@
  \defbibenvironment{bibliography}
    {\list%
       {\MethodFormat}%
       {\setlength{\labelwidth}{\labelnumberwidth}%
        \setlength{\leftmargin}{\labelwidth}%
        \setlength{\labelsep}{\biblabelsep}%
        \addtolength{\leftmargin}{\labelsep}%
        \setlength{\itemsep}{\bibitemsep}%
        \setlength{\parsep}{\bibparsep}}%
        \renewcommand*{\makelabel}[1]{\hss##1}
        }%
    {\endlist}%
  {\item}%
\fi
\makeatother
\begin{document}
\chapter{Introduction}
\addcontentsline{toc}{chapter}{Introduction}

Our society has become heavily reliant on computer systems in order to function. Unexpected behavior of software often results merely in minor inconvenience. However is some cases, it might pose a threat to human safety. Therefore it is crucial to have the ability to verify if the properties of software in question behave as intended. 

One such verification method is model-checking. The basic idea of this method is using finite state automata to accurately describe the current state of the given program at any time during its execution. The states of the automaton represent the current state of the program, and the transitions between them describe how specific events shift the state of the program from one to another. To be able to describe programs that should never terminate, we use $\omega$-automata -- finite automata on infinite words. During the process of model-checking, a product of two automata is created. It is desirable for the product to be as small as possible, which can be achieved by applying reduction methods to the original automata. 

For some reduction techniques, such as simulation-based reduction and degeneralization, it is beneficial to minimize the number of acceptance sets in the automaton. In \cite{spin2013}, the authors introduce the SCC-based simplifications of acceptance conditions. It is a method that proposes how to reduce the number of acceptance sets in Transition-based Büchi automata (TGBA). For each strongly connected component, it evaluates the relation between acceptance sets and removes the useless ones. 

The goal of this thesis is to propose and implement an analogous simplification method for transition-based Emerson-Lei automata (TELA).  It identifies useless acceptance sets in SCCs and removes them. 

\section{Structure of the thesis}
This thesis is structured into several chapter. Chapter two contains the basic definitions of constructs used throughout the thesis. %TODO: 

\chapter{Basic definitions}
\section{Transition-based Emerson-Lei Automaton (TELA)} 
A transition-based Emerson-Lei automaton, as defined in \cite{bloemen2017}, is a tuple $\mathcal{A} = (\Sigma, Q, q_0, \delta, Acc)$, where
\begin{itemize}
  \item $\Sigma$ is an alphabet,
  \item $Q$ is a finite set of states,
  \item $q_0 \in Q$ is the initial state,
  \item $\delta \subseteq Q \times \Sigma \times Q$ is a transition relation,
  \item $Acc$ is a Boolean formula over terms of the form Fin$(T)$ or Inf$(T)$ for any subset $T \subseteq \delta$. %positive Boolean function 
\end{itemize}

For a transition $t = (t^s, t^l, t^d) \in \delta$, $t^s$ denotes its source, $t^l$ its label and $t^d$ its destination.

$\rho \in \delta^\omega$ where $\rho(0)^s = q_0 \wedge \forall i \geq 0 : \rho(i)^d = \rho(i + 1)^s$ is a run of $\mathcal{A}$ over $\Sigma^\omega$.

The acceptance of a run $\rho$ is defined by evaluating $Acc$ over $\rho$ where
\begin{itemize}
  \item Fin$(T)$ is true if all the transitions in $T$ occur finitely often, 
  \item Inf$(T)$ is true if some transition in $T$ occurs infinitely often.
\end{itemize}

\section{Transition-based Generalized Büchi Automaton (TGBA)}
Transition-based Generalized Büchi Automaton, as defined in \cite{bloemen2017}, is a TELA $\mathcal{A'} = (\Sigma, Q, q_0, \delta, Acc)$ where $Acc = \text{Inf}(T_1) \wedge \text{Inf}(T_2) \wedge \dots \wedge \text{Inf}(T_n)$ for some $n$. A run of TGBA over an infinite word is accepting if some transition in $T(i)$ occurs infinitely often for all $i \in \{1, 2, \dots, n\}$. 

\chapter{SCC-based simplification of acceptance conditions of TGBA} 
Let $\mathcal{A'}$ be a TGBA with $n$ acceptance sets: 
\begin{equation*}
  Acc = \text{Inf}(T_1) \wedge \text{Inf}(T_2) \wedge \dots \wedge \text{Inf}(T_n).
\end{equation*}
Let $A_\delta = A_1, \dots, A_m$ be the set of all transitions induced by the accepting SCCs of $\mathcal{A'}$. 
Any accepting run of $\mathcal{A'}$ will be contained in some accepting SCC. Therefore any transitions that are not in $A_\delta$ can be removed without changing the language accepted by $A'$. We can modify the acceptance sets of $\mathcal{A'}$ as follows: 
\begin{equation*}
  Acc_1 = \{ \text{Inf}(T_1) \cap A_\delta \wedge \text{Inf}(T_2) \cap A_\delta \wedge \dots \wedge \text{Inf}(T_n) \cap A_\delta \}.
\end{equation*}
If $Acc_1$ contains such sets that $T_i \subseteq T_j$ and $i \neq j$, we can remove $T_j$, because any run that visits $T_i$ infinitely often also visits $T_j$ infinitely often. Therefore, removing $T_j$ will not change the language.
For all SCCs denoted $\{1, 2, \dots, m\}$ we can define a set of indices of useless acceptance sets
\begin{equation*}
  U_k = \{ j \in \{1, 2, \dots, n\} \mid \exists i \in  \{1, 2, \dots, n\}, (T_i \cap A_k \subsetneq T_j) \vee (T_i = T_j \wedge j > i) \} %TODO: undo zmatek
\end{equation*}
where $k \in \{1, 2, \dots, m\}$. In the definition we can see that if the two sets $T_i$ and $T_j$ are equal only one of them belongs to $Acc_U$. 





\chapter{SCC-based simplification of acceptance conditions of TELA}
%TODO: mention dnf
\section{Simplification of SCCs}
\subsection{Inclusion of Fin-type sets in conjunction}

\section{Merging acceptance conditions of SCCs}

\section{Restoring equivalence with original automaton}


\chapter{Implementation}


\chapter{Experimental Evaluation}


\chapter{Conclusion}



\printbibliography[heading=bibintoc]

\end{document}
