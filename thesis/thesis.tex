%%%%%%%%%%%%%%%%%%%%%%%%%%%%%%%%%%%%%%%%%%%%%%%%%%%%%%%%%%%%%%%%%%%%
%% I, the copyright holder of this work, release this work into the
%% public domain. This applies worldwide. In some countries this may
%% not be legally possible; if so: I grant anyone the right to use
%% this work for any purpose, without any conditions, unless such
%% conditions are required by law.
%%%%%%%%%%%%%%%%%%%%%%%%%%%%%%%%%%%%%%%%%%%%%%%%%%%%%%%%%%%%%%%%%%%%

\documentclass[
  digital, %% This option enables the default options for the
           %% digital version of a document. Replace with `printed`
           %% to enable the default options for the printed version
           %% of a document.
  twoside, %% This option enables double-sided typesetting. Use at
           %% least 120 g/m² paper to prevent show-through. Replace
           %% with `oneside` to use one-sided typesetting; use only
           %% if you don’t have access to a double-sided printer,
           %% or if one-sided typesetting is a formal requirement
           %% at your faculty.
  table,   %% This option causes the coloring of tables. Replace
           %% with `notable` to restore plain LaTeX tables.
  lof,     %% This option prints the List of Figures. Replace with
           %% `nolof` to hide the List of Figures.
  lot,     %% This option prints the List of Tables. Replace with
           %% `nolot` to hide the List of Tables.
  %% More options are listed in the user guide at
  %% <http://mirrors.ctan.org/macros/latex/contrib/fithesis/guide/mu/fi.pdf>.
]{fithesis3}
%% The following section sets up the locales used in the thesis.
\usepackage[resetfonts]{cmap} %% We need to load the T2A font encoding
\usepackage[T1,T2A]{fontenc}  %% to use the Cyrillic fonts with Russian texts.
\usepackage[
  main=english, %% By using `czech` or `slovak` as the main locale
                %% instead of `english`, you can typeset the thesis
                %% in either Czech or Slovak, respectively.
]{babel}        %% foreign texts to be typeset as follows:
%%
%%   \begin{otherlanguage}{german}  ... \end{otherlanguage}
%%   \begin{otherlanguage}{russian} ... \end{otherlanguage}
%%   \begin{otherlanguage}{czech}   ... \end{otherlanguage}
%%   \begin{otherlanguage}{slovak}  ... \end{otherlanguage}
%%
%% For non-Latin scripts, it may be necessary to load additional
%% fonts:
\usepackage{paratype}
%%
%% The following section sets up the metadata of the thesis.
\thesissetup{
    date          = \the\year/\the\month/\the\day,
    university    = mu,
    faculty       = fi,
    type          = bc,
    author        = {Tereza Šťastná},
    gender        = f,
    advisor       = {doc. RNDr. Jan Strejček, Ph.D.},
    title         = {Simplification\\ of Acceptance Condition\\ of Emerson-Lei Automata},
    TeXtitle      = {Simplification\\ of Acceptance Condition\\ of Emerson-Lei Automata},
    keywords      = {TELA, TGBA, omega-automata, SPOT, SCC-based simplifications},
    TeXkeywords   = {TELA, TGBA, $\omega$-automata, SPOT, SCC-based simplifications},
    abstract      = {%
    This thesis proposes SCC-based simplifications of acceptance condition formula for transition-based Emerson-Lei automata (TELA). These simplifications aim to reduce the number of unique acceptance sets of the automaton. The thesis includes a Python program that implements the proposed method. The effectivity of the program is experimentally evaluated.
    },
    thanks        = {%TODO:
      These are the acknowledgements for my thesis, which can

      span multiple paragraphs.
    },
    bib           = bibliography.bib,
}
\usepackage{makeidx}      %% The `makeidx` package contains
\makeindex                %% helper commands for index typesetting.
%% These additional packages are used within the document:
\usepackage{paralist} 
\usepackage{amsmath}  
\usepackage{amsthm}
\usepackage{amssymb}
\usepackage{amsfonts}
\usepackage{url}      
\usepackage{markdown} 
\usepackage{tabularx} 
\usepackage{tabu}
\usepackage{booktabs}
\usepackage{listings} 
\usepackage{tikz}
\usetikzlibrary{automata,nicearrows,bending,myautomata}
\lstset{
  basicstyle      = \ttfamily,
  identifierstyle = \color{black},
  keywordstyle    = \color{blue},
  keywordstyle    = {[2]\color{cyan}},
  keywordstyle    = {[3]\color{olive}},
  stringstyle     = \color{teal},
  commentstyle    = \itshape\color{magenta},
  breaklines      = true,
}
\usepackage{floatrow} %% Putting captions above tables
\floatsetup[table]{capposition=top}
%% The following code fixes the rendering of BibLaTeX ISO 690
%% references in old TeX Live (such as the one at Overleaf).
\thesisload
\makeatletter
\def\thesis@biblatexiso@fix@package{iso-numeric.bbx}
\def\thesis@biblatexiso@fix@end{\relax}
\newif\ifthesis@biblatexiso@fix@
\thesis@biblatexiso@fix@false
\def\thesis@biblatexiso@fix@next#1,{%
  \def\thesis@biblatexiso@fix@current{#1}%
  \ifx\thesis@biblatexiso@fix@current\thesis@biblatexiso@fix@package
    \thesis@biblatexiso@fix@true
  \fi
  \ifx\thesis@biblatexiso@fix@current\thesis@biblatexiso@fix@end
    \expandafter
    \@gobble
  \fi
  \thesis@biblatexiso@fix@next
}
\expandafter\expandafter\expandafter\thesis@biblatexiso@fix@next\@filelist,\relax,
\ifthesis@biblatexiso@fix@
  \defbibenvironment{bibliography}
    {\list%
       {\MethodFormat}%
       {\setlength{\labelwidth}{\labelnumberwidth}%
        \setlength{\leftmargin}{\labelwidth}%
        \setlength{\labelsep}{\biblabelsep}%
        \addtolength{\leftmargin}{\labelsep}%
        \setlength{\itemsep}{\bibitemsep}%
        \setlength{\parsep}{\bibparsep}}%
        \renewcommand*{\makelabel}[1]{\hss##1}
        }%
    {\endlist}%
  {\item}%
\fi
\makeatother
\begin{document}
% Tikz styles 
\tikzset{
  collacc0/.style={fill=blue!50!cyan},
  collacc1/.style={fill=magenta},
  collacc2/.style={fill=orange!90!black},
  collacc3/.style={fill=green!70!black},
  collacc4/.style={fill=blue!50!black},
}

\newcommand{\accmarkblue}{\begin{tikzpicture} \node[accset, collacc0] {}; \end{tikzpicture}}
\newcommand{\accmarkmag}{\begin{tikzpicture} \node[accset, collacc1] {}; \end{tikzpicture}}
\newcommand{\accmarkor}{\begin{tikzpicture} \node[accset, collacc2] {}; \end{tikzpicture}}
\newcommand{\accsquare}{\begin{tikzpicture} \node[accset, collacc0] {}; \end{tikzpicture}}

\chapter{Introduction}
Our society has become heavily reliant on computer systems in order to function. Unexpected behavior of software often results merely in minor inconvenience. However is some cases, it might pose a threat to human safety. Therefore it is crucial to have the ability to verify if the properties of software in question behave as intended. 

One such verification method is model-checking. The basic idea of this method is using finite state automata to accurately describe the current state of the given program at any time during its execution. The states of the automaton represent the current state of the program, and the transitions between them describe how specific events shift the state of the program from one to another. To be able to describe programs that should never terminate, we use $\omega$-automata -- finite automata on infinite words. During the process of model-checking, a product of two automata is created. It is desirable for the product to be as small as possible, which can be achieved by applying reduction methods to the original automata. 

For some reduction techniques, such as simulation-based reduction and degeneralization, it is beneficial to minimize the number of acceptance sets in the automaton. Babiak et al. \cite{spin2013} introduce the SCC-based simplifications of acceptance conditions. It is a method that proposes how to reduce the number of acceptance sets in Transition-based Büchi automata (TGBA). For each strongly connected component, it evaluates the relation between acceptance sets and removes the useless ones. 

The goal of this thesis is to propose and implement an analogous simplification method for transition-based Emerson-Lei automata (TELA). This method includes techniques to identify useless acceptance sets and modifications to the acceptance condition formula. During this process, we obtain separate acceptance condition formulas for each accepting SCC. Therefore, the presented method also describes how to merge these formulas into one. Finally, it describes the necessary modifications which ensure that the simplified automaton and the original automaton are equivalent. The desired outcome is to reduce the number of unique acceptance sets in the automaton. 

This thesis is structured into several chapters. Chapter two contains the basic definitions used throughout the thesis. Chapter three summarizes the ideas of the SCC-based simplification method for TGBA \cite{spin2013}. Chapter four presents the SCC-based simplification of TELA. Chapter five discusses the implementation specifics of this method. Chapter six presents the results of experimental evaluation of the implemented tool. Chapter seven contains the conclusion. TODO: add more details

\chapter{Basic definitions}
\label{chap:basic_definitions}
TODO: popis obsahu kapitoly

\section{Transition-based Emerson-Lei automaton (TELA)} 
\label{sec:tela}
A transition-based Emerson-Lei automaton, as defined in \cite{bloemen2017}, is a tuple $\mathcal{A} = (\Sigma, Q, q_0, \delta, Acc)$, where
\begin{itemize}
  \item $\Sigma$ is an alphabet,
  \item $Q$ is a finite set of states,
  \item $q_0 \in Q$ is the initial state,
  \item $\delta \subseteq Q \times \Sigma \times Q$ is a transition relation,
  \item $Acc$ is the acceptance condition formula, defined as a positive Boolean function over terms of the form Fin$(T_i)$ or Inf$(T_i)$ for any acceptance set $T_i \subseteq \delta, T_i \in \{1, 2, \dots, n\}$. 
  
  In figures, an acceptance set is represented by acceptance marks (for example \accmarkblue) placed on the transitions which belong to that set. 
\end{itemize}

For a transition $t = (t^s, t^l, t^d) \in \delta$, $t^s$ denotes its source, $t^l$ its label and $t^d$ its destination.

$\rho \in \delta^\omega$ where $\rho(0)^s = q_0 \wedge \forall j \geq 0 \colon \rho(j)^d = \rho(j + 1)^s$ is a run of $\mathcal{A}$ over $\Sigma^\omega$.

The acceptance of a run $\rho$ is defined by evaluating $Acc$ over $\rho$ where
\begin{itemize}
  \item Fin$(T_i)$ is true if all the transitions in $T_i$ occur finitely often, 
  \item Inf$(T_i)$ is true if some transition in $T_i$ occurs infinitely often.
\end{itemize}

\section{Transition-based generalized Büchi automaton (TGBA)}
\label{sec:tgba}
Transition-based Generalized Büchi Automaton, as defined in \cite{bloemen2017}, is a TELA $\mathcal{A'} = (\Sigma, Q, q_0, \delta, Acc)$ where 
\begin{equation*}
  Acc = \text{Inf}(T_1) \wedge \text{Inf}(T_2) \wedge \dots \wedge \text{Inf}(T_n)
\end{equation*} 
for some $n$. A run of TGBA over an infinite word is accepting if some transition in $T(i)$ occurs infinitely often for all $i \in \{1, 2, \dots, n\}$. 

\chapter{SCC-based simplification of acceptance conditions of TGBA} 
\label{chap:tgba_simpl}
This chapter summarizes the SCC-based simplifications of TGBA proposed by Babiak et al. \cite{spin2013}.

Let $\mathcal{A'}$ be a TGBA with $n$ acceptance sets: 
\begin{equation*}
  Acc = \text{Inf}(T_1) \wedge \text{Inf}(T_2) \wedge \dots \wedge \text{Inf}(T_n).
\end{equation*}
Let $m$ be the number or accepting SCCs in $\mathcal{A'}$ and $A_\delta = \{A_1, \dots, A_m\}$ be the set of all transitions induced by the accepting SCCs of $\mathcal{A'}$. 
Any accepting run of $\mathcal{A'}$ will be contained in some accepting SCC. Therefore any transitions that are not in $A_\delta$ can be removed without changing the language accepted by $A'$. We can modify the acceptance sets of $\mathcal{A'}$ as follows: 
\begin{equation*}
  Acc = \text{Inf}(T_1 \cap A_\delta) \wedge \text{Inf}(T_2 \cap A_\delta) \wedge \dots \wedge \text{Inf}(T_n \cap A_\delta).
\end{equation*}
If $Acc$ contains such sets that $T_i \subseteq T_j$ and $i \neq j$, we can remove $T_j$, because any run that visits $T_i$ infinitely often also visits $T_j$ infinitely often. Therefore, removing $T_j$ will not change the language.
For each accepting SCC we can define a set of indices of useless acceptance sets 
\begin{equation*}
  \begin{aligned}
    U_k = &\{ j \in \{1, 2, \dots, n\} \mid \exists i \in \{1, 2, \dots, n\}, \\
    &(T_i \cap A_k \subsetneq T_j \cap A_k) \vee (T_i \cap A_k = T_j \cap A_k \wedge j > i)\}.
  \end{aligned}
\end{equation*}
If the two sets $T_i$ and $T_j$ are equal, then, by this definition, only one of their indices belongs to $U_k$, ensuring that one of the sets will be preserved. The set of indices of needed acceptance sets for any accepting SCC is $N_k = \{1, 2, \dots, n\} \smallsetminus U_k$. 

This may result in each SCC having a different number of acceptance sets, however the automaton as a whole can only have one set of acceptance sets. The number of acceptance sets needed in the simplified automaton is $n' = max_{k \in \{1, 2, \dots, n\}} |N_k|$. Let $N'_k$ be a copy of $N_k$. By adding $n' - |N_k|$ indices from $U_k$ to $N'_k$, we ensure that $|N'_k| = n'$ for each accepting SCC. Let $\alpha_k \colon \{1, \dots, n'\} \to N'_k$ be any bijection. Now we can define the new acceptance sets of the simplified automaton as
\begin{equation*}
  T'_i = \underset{k \in \{1, \dots, m\}}\bigcup (T_{\alpha_k(i)} \cap A_k)
\end{equation*}
in acceptance condition formula $Acc = \text{Inf}(T'_1) \wedge \text{Inf}(T'_2) \wedge \dots \wedge \text{Inf}(T'_{n'})$.

\begin{figure}[h]
  \begin{center}
    \begin{tikzpicture}[->,>=stealth,shorten >=1pt,very thick,initial text={}]
      \tikzstyle{every state}=[very thick]
      \node[state, initial above]   (0) at (0,0) {0};
      \node[state]            (1) at (-3, 0) {1};
      \node[state]            (2) at (3, 0) {2};

      \path[->, very thick]
              (0) edge[bend right]  node [above] {\texttt{tt}} (1) 
                  edge[bend left]   node [above] {\texttt{tt}} (2) 
              (1) edge[loop above]  node[above] {$ab$} node[accset, collacc1, pos=0.40] {} node[accset, collacc0, pos=0.15] {} node[accset, collacc2, pos=0.75] {} (1) 
                  edge[loop left]   node[left] {$a\overline{b}$} node[accset, collacc1] {} node[accset, collacc2, pos=0.15] {} (1) 
                  edge[loop right]  node[right] {$\overline{a}\overline{b}$} node[accset, collacc1] {} (1) 
                  edge[loop below]  node[below] {$\overline{a}b$} node[accset, collacc1] {} node[accset, collacc0, pos=0.15] {} (1) 
              (2) edge[loop above]  node[above] {$c$} node[accset, collacc1, pos=0.40] {} node[accset, collacc0, pos=0.15] {} node[accset, collacc2, pos=0.75] {} (2) 
                  edge[loop below]  node[below] {$\overline{c}$} node[accset, collacc0] {} node[accset, collacc2, pos=0.15] {} (2);    
    \end{tikzpicture}
    \caption{TGBA with three acceptance sets}
    \label{fig:tgba}
  \end{center}
\end{figure}

In Figure \ref{fig:tgba} there is a TGBA with two accepting SCCs. Let $S_1$ be the SCC comprised of state 1 and $S_2$ be the SCC comprised of state 2. The set of useless acceptance sets for $S_1$ is $U_1 = \{\accmarkmag\}$, because $\accmarkblue \subseteq \accmarkmag$ (and also $\accmarkor \subseteq \accmarkmag$). For $S_2$ it is $U_2 = \{\accmarkblue, \accmarkor\}$, because $\accmarkmag \subseteq \accmarkblue$ and $\accmarkmag \subseteq \accmarkor$.

The sets of needed acceptance sets for $S_1, S_2$ are $N_1 = \{\accmarkblue, \accmarkor\}$, $N_2 = \{\accmarkmag\}$ respectively. The number of needed acceptance sets for the whole automaton is $n' = 2$. We can define $N'_1 = N_1, N'_2 = N_2 \cup \{\accmarkor\}, \alpha_1 (\accmarkor) = \accmarkor, \alpha_1 (\accmarkblue) = \accmarkblue, \alpha_2 (\accmarkor) = \accmarkor, \alpha_2 (\accmarkblue) = \accmarkmag$. The result is the TGBA in Figure \ref{fig:simpl_tgba}.

\begin{figure}[h]
    \begin{center}
    \begin{tikzpicture}[->,>=stealth,shorten >=1pt,very thick,initial text={}]
      \tikzstyle{every state}=[very thick]
      \node[state, initial above]   (0) at (0,0) {0};
      \node[state]            (1) at (-3, 0) {1};
      \node[state]            (2) at (3, 0) {2};

      \path[->, very thick]
              (0) edge[bend right]  node [above] {\texttt{tt}} (1) 
                  edge[bend left]   node [above] {\texttt{tt}} (2) 
              (1) edge[loop above]  node[above] {$ab$} node[accset, collacc0, pos=0.15] {} node[accset, collacc2] {} (1) 
                  edge[loop left]   node[left] {$a\overline{b}$} node[accset, collacc2] {} (1) 
                  edge[loop right]  node[right] {$\overline{a}\overline{b}$} (1) 
                  edge[loop below]  node[below] {$\overline{a}b$} node[accset, collacc0] {} (1) 
              (2) edge[loop above]  node[above] {$c$} node[accset, collacc0, pos=0.15] {} node[accset, collacc2] {} (2) 
                  edge[loop below]  node[below] {$\overline{c}$} node[accset, collacc2] {} (2);    
    \end{tikzpicture}
    \caption{TGBA from Figure \ref{fig:tgba} after SCC-based acceptance simplification}
    \label{fig:simpl_tgba}
  \end{center}
\end{figure}


\chapter{SCC-based simplification of acceptance conditions of TELA}
\label{chap:simpl_tela}
This chapter presents the proposed simplification methods for TELA. The first section covers the removal of useless transitions from all acceptance sets. The second section describes the process of analyzing SCCs. The third section proposes several techniques that may remove useless acceptance sets from SCCs selected based on the analysis in Section \ref{sec:init_analysis}. Section four presents the method used for merging the separate acceptance condition formulae of all SCCs into one. The resulting formula is the acceptance condition of the simplified automaton. This section also covers how to identify and resolve possible conflicts caused by dependencies between disjuncts. Section five describes the necessary modifications to the simplified automaton that ensure its equivalence with the input automaton. 

In this chapter, let the input TELA have at least one acceptance set. Let the acceptance condition formula of the TELA be in disjunctive normal form (DNF). When referring to the whole automaton, we use notation $\mathcal{A}$. $Acc$ denotes the acceptance condition formula of the whole automaton. $\mathcal{A'}$ refers to a sub-automaton induced by an SCC of the original automaton, and $Acc'$ refers to the acceptance condition formula of $\mathcal{A'}$.

Because the formula is in DNF, we can represent it as 
\begin{equation*}
  Acc = \{d_1, d_2, \dots, d_m\},
\end{equation*}
where each $d_k$ is a disjunct. Further, each such disjunct $d_k$ is represented as a set of conjuncts. For example, formula $Acc_e = (\text{Inf}(T_1) \wedge \text{Fin}(T_2)) \vee (\text{Inf}(T_3))$ is represented by $Acc_e = \{d_1, d_2\}$, where $d_1 = \{\text{Inf}(T_1), \text{Fin}(T_2)\}, d_2 = \{\text{Inf}(T_3)\}$. 

Let $S$ be the set of states of an SCC. Then $A = \{t \in \delta \mid t^s \in S \wedge t^d \in S\}$ is the set of all transitions of that SCC. In other words, $A$ is a set of all transitions in $\mathcal{A'}$.

\section{Removing useless transitions from acceptance sets}
Any accepting run of $\mathcal{A}$ is always contained within an accepting SCC. Let $T_{o}$ be a set of all transitions outside of accepting SCCs. Such transitions can be removed from all acceptance sets without changing the language, because transitions outside of accepting SCCs are always visited finitely often by any run of $\mathcal{A}$. Therefore, such transition $t_{o}$ never changes the boolean value of set $T_i$ such that $t_{o} \in T_i, i \in \{1, 2, \dots, n\}$ in either of Inf$(T_i)$, Fin$(T_i)$.

We can modify each acceptance set $T_i, i \in \{1, 2, \dots, n\}$ of $Acc$ as follows:
\begin{equation*}
  T_i = T_i \smallsetminus T_{o}.
\end{equation*}

\section{Initial analysis of SCCs}
\label{sec:init_analysis}
Before applying any simplification techniques, we first analyze each SCC and assign it one of the following attributes: \emph{always accepting}, \emph{never accepting}, \emph{maybe accepting}. The purpose of this classification is to help determine the necessary modifications when restoring equivalence with the original automaton as described in Section \ref{sec:restore_equiv}. 

Let $T_{all} = \{i \in \{1, 2, \dots, n\} \mid \forall t \in A \colon t \in T_i \}$ be the set of indices of acceptance sets that contain all transitions of the SCC.

Let $T_{any} = \{i \in \{1, 2, \dots, n\} \mid \exists t \in A \colon t \in T_i \}$ be the set of indices of acceptance sets that contain at least one transition of the SCC.

Let $Acc'$ be a copy of $Acc$. Let $T_i, i \in \{1, 2, \dots, n\}$ be an acceptance set of $\mathcal{A}$. Now we perform the following substitutions for each disjunct $d_k$ in $Acc'$. TODO: prepis
\begin{itemize}
  \item If $i \in T_{all}$ then we substitute \emph{true} for Inf$(T_i)$.
  \item If $i \in T_{all}$ then we substitute \emph{false} for Fin$(T_i)$.
  \item If $i \notin T_{any}$ then we substitute \emph{false} for Inf$(T_i)$.
  \item If $i \notin T_{any}$ then we substitute \emph{true} for Fin$(T_i)$.
\end{itemize}

Any disjunct in $Acc'$ that only contains \emph{true} is evaluated as \emph{true}. Any disjunct in $Acc'$ that contains at least one \emph{false} is evaluated as \emph{false}. 

If any disjunct in $Acc'$ equals \emph{\{true\}}, then the SCC is \emph{always accepting}. If all disjuncts in $Acc'$ are equal to \emph{\{false\}}, then the SCC is \emph{never accepting}. If neither of the above conditions applies, then the SCC is \emph{maybe accepting}.

\section{SCC simplifications}
This section presents the simplification methods that can be applied to all SCCs, which are \emph{maybe accepting}. Using these methods on \emph{always accepting} or \emph{never accepting} SCCs is not necessary because the acceptance formulae of such SCCs can be replaced with \emph{true} or \emph{false}, respectively. 

Unlike TGBA, TELA have complex acceptance condition formulae. Therefore, to identify which acceptance sets are useless, we need to consider not only the relation between the transitions in these sets but also the specifics of their occurrence in the acceptance condition formula. 

For example, consider acceptance sets $T_1, T_2, T_3, T_1 \subseteq T_2$ and the formula $Acc_a = \text{Inf}(T_1) \wedge \text{Inf}(T_2)$. In this case, we can apply the simplifications described in Chapter \ref{chap:tgba_simpl} and remove $T_2$, because the form of $Acc_a$ is identical with the form of TGBA formula. However, in $Acc_b = \text{Inf}(T_1) \vee (\text{Inf}(T_2) \wedge \text{Fin}(T_3))$ removing $T_2$ could change the language. This example demonstrates that for SCC-based TELA simplifications, it is necessary to carefully define the requirements about the form of the acceptance condition formula.

\subsection{Acceptance condition formula modifications}
In the previous section, we proposed several substitutions used to analyze the SCCs. By using similar approach, we can remove useless disjuncts from $Acc$, and useless conjuncts from the remaining disjuncts.

Let $T_i$ be an acceptance set that only appears in $Acc'$ as Fin$(T_i)$. If all transitions of $\mathcal{A'}$ belong to $T_i$, then during a run of $\mathcal{A'}$ at least one transition of $T_i$ is visited infinitely often. Therefore Fin$(T_i)$ is always \emph{false} for any run of $\mathcal{A'}$. Any disjunct $d_k$ of $Acc'$ that contains Fin$(T_i)$ can be removed without changing the language.

More formally, if the following conditions hold
\begin{itemize}
  \item $i \in \{1,2, \dots, n\}$,
  \item $k\in \{1,2, \dots, m\}$,
  \item $T_i = A$,
\end{itemize}
then we can define the set of useless disjuncts as
\begin{equation*}
  U = \{d_k \in Acc \mid \text{Fin}(T_i) \in d_k \}
\end{equation*}
and remove it from $Acc'$:
\begin{equation*}
  Acc' = Acc' \smallsetminus U.
\end{equation*}

TODO: priklad

Let $T_i$ be an acceptance set that only appears in $Acc'$ as Inf$(T_i)$. If $T_i$ does not contain any transitions of $\mathcal{A'}$, then any disjunct $d_k$ in $Acc'$ that contains Inf$(T_i)$ can be removed. Removing $d_k$ does not change the language, because any run of $\mathcal{A'}$ does not visit any transitions that belong to $T_i$; therefore, Inf$(T_i)$ is never satisfied.

Formally, if these conditions are met
\begin{itemize}
  \item $i \in \{1,2, \dots, n\}$,
  \item $k\in \{1,2, \dots, m\}$,
  \item $T_i \cap A = \emptyset$,
\end{itemize}
then 
\begin{equation*}
  U = \{d_k \in Acc \mid \text{Inf}(T_i) \in d_k \}
\end{equation*}
is the set of useless disjuncts that can be removed from $Acc'$:
\begin{equation*}
  Acc' = Acc' \smallsetminus U.
\end{equation*}

TODO: priklad

If an acceptance set $T_i$ contains all transitions of the SCC, then Inf$(T_i)$ is always evaluated as \emph{true}. Therefore Inf$(T_i)$ can be removed from all disjuncts of $Acc'$ without changing the language. 
In other words, we can define a set of useless conjuncts 
\begin{equation*}
  U = \{\text{Inf}(T_i) \mid i \in \{1,2, \dots, n\} \wedge T_i = A\}.
\end{equation*}
Then we can remove the useless conjuncts in $U$ from each disjunct of $Acc'$ as follows:
\begin{equation*}
  Acc' = \{d_1 \smallsetminus U, d_2 \smallsetminus U, \dots, d_m \smallsetminus U\}.
\end{equation*}

Similarly, Fin$(T_i)$ is always true if no transitions of the SCC belong to $T_i$. Therefore all such Fin$(T_i)$ can be removed from $Acc'$ without changing the language.
The set of useless conjuncts can be defined as:
\begin{equation*}
  U = \{\text{Fin}(T_i) \mid i \in \{1,2, \dots, n\} \wedge T_i \cap A = \emptyset\}.
\end{equation*}
Then the $Acc'$ can be modified as follows:
\begin{equation*}
  Acc' = \{d_1 \smallsetminus U, d_2 \smallsetminus U, \dots, d_m \smallsetminus U\}.
\end{equation*}

Let $d_k, d_l$ be disjuncts of $Acc'$ such that $d_k \subseteq d_l$. In this case, disjunct $d_l$ can be removed from $Acc'$, because either $d_k$ is satisfied and the whole formula is evaluated as \emph{true} regardless of $d_l$, or $d_k$ is not satisfied, in which case $d_l$ is also not satisfied. Therefore, removing $d_l$ does not change the language.

More formally, if we define the set of useless disjuncts as:
\begin{equation*}
  U = \{d_l \in Acc' \mid \exists d_k \in Acc' \colon d_k \subseteq d_l\},
\end{equation*}
then we may remove these useless disjuncts from $Acc'$ as follows:
\begin{equation*}
  Acc' = Acc' \smallsetminus U.
\end{equation*}

\subsection{Inclusion-based simplifications}
%Inf & Inf
We can modify the simplifications from Chapter \ref{chap:tgba_simpl} to make them viable for TELA by adding a prerequisite that demands a specific form of the acceptance formula. Let $T_1 \subseteq T_2$ be two distinct acceptance sets that appear in the acceptance condition formula only in terms Inf$(T_1)$ and Inf$(T_2)$. Let $Acc'$ be in such form that in any disjunct of the formula, either both Inf$(T_i), \text{Inf}(T_j)$ are present, or neither is present. If the required conditions are met, we may remove $T_2$ from the formula. 

More formally, if the following conditions hold
\begin{itemize}
  \item $i, j \in \{1, 2, \dots, n\} \wedge i \neq j$,
  \item $T_i \subsetneq T_j \vee (T_i = T_j \wedge j > i)$,
  \item $\forall d_k \in Acc' \colon (\text{Inf}(T_i) \in d_k \Leftrightarrow \text{Inf}(T_j) \in d_k)$,
\end{itemize}
then we can remove Inf$(T_j)$ from $Acc'$: 
\begin{equation*}
  Acc' = \{d_1 \smallsetminus \{\text{Inf}(T_j)\}, d_2 \smallsetminus \{\text{Inf}(T_j)\}, \dots, d_m \smallsetminus \{\text{Inf}(T_j)\}\}.
\end{equation*} 

TODO: priklad

%Fin & Fin
Let $T_1 \subseteq T_2$ be two distinct acceptance sets that only appear in $Acc'$ as Fin$(T_i)$ and Fin$(T_j)$. If each disjunct in $Acc'$ either contains both Fin$(T_i)$ and Fin$(T_j)$ or contains neither of them, then Fin$(T_i)$ can be removed from $Acc'$.  Removing Fin$(T_i)$ does not change the language, because when transitions in $T_j$ are visited finitely often during a run of $\mathcal{A'}$, then also transitions in $T_i$ are visited finitely often. 

In other words, if these conditions are met:
\begin{itemize}
  \item $i, j \in \{1,2, \dots, n\} \wedge i \neq j$,
  \item $T_i \subsetneq T_j \vee (T_i = T_j \wedge j > i)$,
  \item $\forall d_k \in Acc' \colon (\text{Fin}(T_i) \in d_k \Leftrightarrow \text{Fin}(T_j) \in d_k)$,
\end{itemize}
then Fin$(T_i)$ can be removed from $Acc'$:
\begin{equation*}
  Acc' = \{d_1 \smallsetminus \{\text{Fin}(T_i)\}, d_2 \smallsetminus \{\text{Fin}(T_i)\}, \dots, d_m \smallsetminus \{\text{Fin}(T_i)\}\}.
\end{equation*}

TODO: priklad

%Inf & Fin
Let $T_1 \subseteq T_2$ be two distinct acceptance sets that only appear in $Acc'$ as Inf$(T_i)$ and Fin$(T_j)$. If there is such disjunct $d_k$ of $Acc'$, that it contains both Inf$(T_i)$ and Fin$(T_j)$, then $d_k$ can be removed from $Acc'$. Removing $d_k$ does not change the language, because any run of $\mathcal{A'}$ that visits $T_j$ finitely often cannot visit any transition in $T_i$ infinitely often. Therefore $d_k$ is evaluated as \emph{false} for any run of $\mathcal{A'}$. 

If the following conditions hold
\begin{itemize}
  \item $i, j \in \{1, 2, \dots, n\} \wedge i \neq j$,
  \item $T_i \subsetneq T_j \vee (T_i = T_j \wedge j > i)$,
\end{itemize}
then the we can define the set of useless disjuncts as:
\begin{equation*}
  U = \{d_k \in Acc' \mid \text{Inf}(T_i) \in d_k \wedge \text{Fin}(T_j) \in d_k \}.
\end{equation*}
$Acc'$ without these useless sets is defined as:
\begin{equation*}
  Acc' = Acc' \smallsetminus U.
\end{equation*}

TODO: priklad

\subsection{Simplification of complementary acceptance sets}
Let $T_i$ and $T_j$ be two distinct acceptance sets that appear in $Acc'$ only as Fin$(T_i)$ and Inf$(T_j)$. Let $T_i$ and $T_j$ be complementary. If each $d_k$ of $Acc'$ that contains Inf$(T_j)$ also contains Fin$(T_i)$, then we can remove Inf$(T_j)$ without changing the language. If a run of $\mathcal{A'}$ is contained in the SCC and all transitions in $T_i$ are visited finitely often, then necessarily at least one transition in $T_j$ is visited infinitely often. Therefore keeping only the conjunct Fin$(T_i)$ in $d_k$ is sufficient. 

More formally, if these conditions are met:
\begin{itemize}
  \item $i, j \in \{1, 2, \dots, n\} \wedge i \neq j$,
  \item $\exists T_i, T_j \colon A \smallsetminus T_i = T_j$,
  \item $\forall d_k \in Acc' \colon \text{Inf}(T_j) \in d_k \Rightarrow \text{Fin}(T_j) \in d_k$,
\end{itemize}
then Inf$(T_j)$ can be removed from $Acc'$:
\begin{equation*}
  Acc' = \{d_1 \smallsetminus \{\text{Inf}(T_j)\}, d_2 \smallsetminus \{\text{Inf}(T_j)\}, \dots, d_m \smallsetminus \{\text{Inf}(T_j)\} \}
\end{equation*}

\subsection{Combining Fin conjuncts}
If $\mathcal{A'}$ has such acceptance sets $T_i, T_j$ and $i \neq j$, that both sets only appear in $Acc'$ as Fin$(T_i)$ and Fin$(T_j)$ and each disjunct of $Acc'$ either contains both of them or neither of them, then we can replace these sets by a new acceptance set. The new acceptance set $T_x$ contains all transitions from both $T_i$ and $T_j$. Fin$(T_i)$ and Fin$(T_j)$ can be removed from all disjuncts, that they belong to, and Fin$(T_x)$ is added to those disjuncts. This modification does not change the language, because Fin$(T_x)$ is equivalent to the expression $(\text{Fin}(T_i) \wedge \text{Fin}(T_j))$, if $T_x$ and $(T_i \cup T_j)$ are equal sets. Either expression is satisfied only if all transitions in $T_x = (T_i \cup T_j)$ are visited finitely often.

In other words, if the following conditions are met
\begin{itemize}
  \item $i, j \in \{1, 2, \dots, n\} \wedge i \neq j$,
  \item $\forall d_k \in Acc' \colon \text{Fin}(T_i) \in d_k \Leftrightarrow \text{Fin}(T_j) \in d_k$,
\end{itemize}
then we can define a new acceptance set $T_x = (T_i \cup T_j)$ and modify $Acc'$ as follows: 
\begin{align*}
  Acc' =& \{d_k \in Acc' \mid \text{Fin}(T_i) \notin d_k\} \cup \\ &\{(d_k \smallsetminus \{\text{Fin}(T_i), \text{Fin}(T_j)\}) \cup \{\text{Fin}(T_x)\} \mid \text{Fin}(T_i) \in d_k\}.
\end{align*}

\section{Merging acceptance condition formulae of all SCCs}
After applying simplification techniques from the previous chapter, each SCC has its own acceptance formula. This section proposes one possible approach to merging acceptance condition formulae. The resulting formula is the acceptance condition formula of the simplified automaton $\mathcal{A}_s$.

This method only processes formulae of \emph{maybe accepting} SCCs. The formulae of \emph{never accepting} and \emph{always accepting} SCCs are equivalent to expressions \emph{false} and \emph{true}, respectively. These expressions can merge with any formula without it requiring any modifications. Therefore, it is not necessary to include them in the merging process. 

Let $C = \{Scc_1, Scc_2, \dots, Scc_z\}$ be the set of all \emph{maybe accepting} SCCs of $\mathcal{A}$. The set of acceptance condition formulae of these SCCs is $F = \{Acc_1, Acc_2, \dots, Acc_z\}$. Let $F$ be a sorted set with the formulae ordered by their cardinality in descending order, meaning that the formula with the most disjuncts is first. Let $C$ be sorted so that the SCCs match the order of their respective formulae in $F$. In other words, the first formula in $F$, $Acc_1$, belongs to $Scc_1$.

We select $Acc_1$, the formula with the highest cardinality, to be the base for the new acceptance condition formula $Acc_s$. Let $Acc_s$ be a copy of $Acc_1$ and let $Scc_s$ be an alias for $Scc_1$. Then we can remove $Acc_1$ from $F$ and $Scc_1$ from $C$. This method modifies $Acc_s$ so that we may define an injective function for any formula $Acc_k \in F$ that maps the acceptance sets of $Acc_k$ to the sets of $Acc_s$. 

Dependency occurs when a formula contains the same conjunct in two or more disjuncts. In the process of merging formulae, conflicts caused by dependencies may arise and lead to the language accepted by the automaton being changed. Resolving these conflicts is described in subsection \ref{subsec:resolve_conflicts}. 

\subsection{Merging acceptace condition formulae}
Merging formulae consists of merging each formula in $C$ with $Acc_s$. When the process is complete the modified $Acc_s$ become the acceptance formula of the whole automaton $\mathcal{A}_s$.

Two formulae are merged by pairing their disjuncts and then pairing the conjuncts in these disjuncts. The pairing of disjuncts is selected based on the lowest cost, meaning we choose the pairing that requires adding the fewest new conjuncts to $Acc_s$. 

Determining the lowest cost pairing is a form of assignment problem. There are several possible solutions. Our implementation uses the Hungarian method \cite{hungarian_method}. The input for this method is matrix $M$. Each element in $M$ represents the cost of merging two disjuncts, $d_k \in Acc_k$ and $d_s \in Acc_s$. The cost is calculated as the amount of Fin\footnote{Here Fin refers to any conjuncts Fin$(T_i) \in d_k$, where $T_i$ is arbitrary.} conjuncts in $d_k$ subtracted from the amount of Fin conjuncts in $d_s$ plus the amount of Inf\footnote{Analogous to Fin notation.} conjuncts in $d_k$ subtracted from the amount of Inf conjuncts in $d_s$. The cost cannot be lower than 0. The rows of $M$ represent the disjuncts of $Acc_s$ and the colums represent the disjuncts of $Acc_k$.

Intuitively, the cost describes how many conjuncts fewer $d_s$ has compared to $d_k$ (Fin conjuncts, and Inf conjuncts are counted separately). For example, if $d_k = \{\text{Fin}(T_1), \text{Inf}(T_2), \text{Inf}(T_3)\}$ is a disjunct of $Acc_k$ and $d_s = \{\text{Inf}(T_2)\}$ is a disjunct of $Acc_s$, then the cost of pairing such disjuncts is 2, because $d_s$ is missing one Inf conjunct and one Fin conjunct. 

\subsection{Merging two disjuncts}
The Hungarian method determines that we pair disjuncts $d_s$ and $d_k$, where $d_s$ is a disjunct of $Acc_s$ and $d_k$ is a disjunct of $Acc_k$. The merging process consists of the following steps:
\begin{enumerate}
  \item Label all conjuncts of $d_s$ as unused.
  \item Try to pair each conjunct in $Acc_k$ with a conjunct in $Acc_s$. Such pairing is possible under these conditions: The conjunct that belongs to $d_k$ is labeled as unused and both conjuncts are of the same type, meaning that either both are Fin or both are Inf.
  \item \begin{enumerate}[a)]
    \item If such pairing exists, then we can define injective function $\varphi_k (T_k) = T_i$, where $T_k$ is the acceptance set of the conjunct in $d_k$, and $T_i$ is the acceptance set of the conjunct in $d_s$. Then label the paired conjunct of $d_s$ as used.
    \item If such pairing does not exist, then add a new conjunct to $d_s$. Let the new conjunct be of the same type as the conjunct in $d_k$ that could not be paired. It is necessary to prevent changing the language when adding the new conjunct. That can be achieved if the conjunct is satisfied by any run of $\mathcal{A'}$. In other words, if its type is Fin, then let the new acceptance set $T_x$, where $x = n + 1$, be an empty set.\footnote{It is necessary to update the number acceptance sets of $\mathcal{A}_s$ to reflect this change. Let $T_i, i \in \{1,2, \dots, n, n+1\}$ denote any acceptance set of $Acc_s$. Further references to $n$ denote the highest number of the set.} If the type is Inf, then let $T_x$ be a set of all transitions of the SCC. Label the new conjunct as used.
  \end{enumerate}
\end{enumerate}
The process is complete when all conjuncts of $d_k$ are successfully paired.

Note that we never need to add any new disjuncts to  $Acc_s$, because it is the formula with the highest cardinality.

To give an example, consider $d_s = \{\text{Inf}(T_1)\}$, $d_k = \{\text{Inf}(T_2), \text{Inf}(T_3)\}$. First step is to mark Inf$(T_1)$ as unused. Then we attempt to find pairing for Inf$(T_2)$. The only possible pairing is with Inf$(T_1)$. Therefore, we can define $\varphi_k(T_2) = T_1$ and label Inf$(T_1)$ as used. Inf$(T_2)$ cannot be paired with any element of $d_s$. It is necessary to add a new conjunct to $d_s$, resulting in $d_s = \{\text{Inf}(T_1), \text{Inf}(T_4)\}$, where $T_4$ is a set of all transition of the SCC. We can define $\varphi_k(T_3) = T_4$ and label Inf$(T_4)$ as used. All conjuncts in $d_k$ are paired, therefore the merging process of disjuncts $d_s$ and $d_k$ is complete.

\subsection{Resolving conflicts caused by dependencies}
\label{subsec:resolve_conflicts}
Before merging the formulae, it is necessary to check if any conflicts may occur during the process. Conflicts during merging are situations that, if unresolved, may lead to two distinct acceptance sets that appear in $Acc_k \in F$ being mapped by $\varphi_k$ to only one set in $Acc_s$. For example, consider $Acc_s = (\text{Inf}(T_1) \wedge \text{Fin}(T_2)) \vee (\text{Inf}(T_1))$, and $Acc_k = \text{Inf}(T_3) \vee \text{Inf}(T_4)$. In this case, if we define $\varphi_k (T_3) = T_1, \varphi_k(T_4) = T_1$, then sets $T_3$ and $T_4$ are both mapped to the same set, which changes the language.   

Conflicts only may occur when $Acc_s$ contains two or more disjuncts that contain the same element. If $Acc_s$ does not contain such disjuncts, then we may proceed with merging the formulae. 
 
If we discover possible conflicts, then we simulate the merging process to determine if any sets are mapped incorrectly. Incorrect mapping occurs in the following situations:
\begin{enumerate}[a)]
  \item Two distinct sets of $Acc_k$ are mapped to one set of $Acc_s$.
  \item There exists a set $T_i$, that appears in $Acc_s$, and an injective function is defined that maps $T_k$ of $Acc_k$ to $T_i$. No set is mapped to the other occurrence of $T_i$. \\
  For example, $Acc_s = \{\{\text{Inf}(T_1), \text{Inf}(T_2)\}, \{\text{Inf}(T_1), \text{Fin}(T_3)\}\}$ and $Acc_k = \{\{\text{Inf}(T_4), \text{Inf}(T_5)\}, \{\text{Fin}(T_6)\}\}$. In this case, we define $\varphi_k (T_4) = T_1, \varphi_k (T_5) = T_2, \varphi_k (T_6) = T_3$, however in the second disjunct of $Acc_s$ the conjunct Inf$(T_2)$ remains unused.
  \item There exists a set $T_i$, that appears in $Acc_s$, and an injective function is defined that maps $T_k$ of $Acc_k$ to $T_i$. The other occurrence of set $T_i$ is in such disjunct of $Acc_s$, which is not paired with any disjunct of $Acc_k$. \\
  For example, $Acc_s = \{\{\text{Inf}(T_1), \text{Inf}(T_2)\}, \{\text{Inf}(T_1)\}$ and $Acc_k = \{\{\text{Inf}(T_4), \text{Inf}(T_5)\}\}$. Similarly to the previous case, the second Inf$(T_1)$ remains unused. 
\end{enumerate}

These incorrect mappings can be avoided by replacing all additional occurrences of the same set with a new set. The new set has to contain the same transitions as the original set, to prevent changing the language. 

More formally, let $T_i$ denote an acceptance set that only appears in $Acc_s$ as Inf$(T_i)$. If the following condition holds
\begin{equation*}
  \exists d_k, d_l \in Acc_s \colon i \in \{1, 2, \dots, n\} \wedge \text{Inf}(T_i) \in d_l \wedge \text{Inf}(T_i) \in d_k \wedge k < l
\end{equation*}
then we modify each disjunct $d_k$ as follows:
\begin{equation*}
  d_k = (d_k \smallsetminus \{\text{Inf}(T_i)\}) \cup \{\text{Inf}(T_x)\}
\end{equation*}
where $T_x = T_i$ and $x = n + 1$. The algorithm is analogous for Fin$(T_i)$.

\section{Restoring equivalence with the original automaton}
\label{sec:restore_equiv}
$Acc_s$ is the acceptance condition formula of the whole automaton $\mathcal{A}_s$. However, the process of SCC-based simplifications resulted in each SCC  having its own distinct formula and acceptance sets. Replacing such formula with $Acc_s$ may change the language. Therefore, it is necessary to modify $Acc_s$ to preserve the acceptance conditions in each SCC. 

$Scc_s$ does not require any modifications, because its acceptance condition formula was selected as the base for $Acc_s$.

To restore the equivalence with the original automaton in \emph{never accepting} SCCs, it is necessary to ensure that each disjunct of $Acc_s$ is evaluated as \emph{false} for any run of $\mathcal{A}_s$. Any disjunct containing Inf$(T_i)$, where $T_i$ is any acceptance set of $Acc_s$, already satisfies this requirement. If there is a disjunct in $Acc_s$ that only contains Fin$(T_i)$ conjuncts, where $T_i$ is any acceptance set of $Acc_s$, then one of its acceptance sets has to be modified as follows:
\begin{equation*}
  T_i = T_i \cup A_{Scc},
\end{equation*}
where $A_{Scc}$ is a set of all transitions that belong to the \emph{never accepting} SCC. Then any run of $\mathcal{A}_s$ that is contained within that SCC is not satisfied because at least one transition of $T_i$ is visited infinitely often; therefore, Fin$(T_i)$ is evaluated as \emph{false}.

For all \emph{always accepting} SCCs, we need to ensure that at least one disjunct of $Acc_s$ is true for any run of $\mathcal{A}_s$ that is contained within such SCC. If there is a disjunct in $Acc_s$ that only contains Fin$(T_i)$, where $T_i$ is any acceptance set of $Acc_s$, then the requirement is satisfied. Otherwise, we choose a disjunct that contains the fewest Inf conjuncts and modify the acceptance sets of $Acc_s$ so that these conjuncts are true for any run of $Acc_s$ that is contained within the \emph{always accepting} SCC. All acceptance sets $T_i$ that appear in that disjunct in conjuncts Inf$(T_i)$ need to be modified as follows:
\begin{equation*}
  T_i = T_i \cup A_{Scc},
\end{equation*}
where $A_{Scc}$ is a set of all transitions that belong to the \emph{never accepting} SCC. Then any run of $\mathcal{A}_s$ that is contained within that SCC is evaluated as \emph{true} because for each acceptance set, $T_i$ that appears in the disjunct as Inf$(T_i)$, at least one transition is visited infinitely often; therefore, any conjuncts in the form of Fin$(T_i)$, where $T_i$ is any acceptance set of $A_s$, do not affect the satisfiability of the disjunct, because any such set $T_i$ does not contain any transitions of the SCC.

\chapter{Implementation}

\chapter{Experimental Evaluation}

\chapter{Conclusion}

\printbibliography[heading=bibintoc]

\makeatletter\thesis@blocks@clear\makeatother
\phantomsection
\addcontentsline{toc}{chapter}{\indexname} 
\printindex

\appendix
\chapter{An appendix}
Here you can insert the appendices of your thesis.

\end{document}
