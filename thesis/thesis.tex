%%%%%%%%%%%%%%%%%%%%%%%%%%%%%%%%%%%%%%%%%%%%%%%%%%%%%%%%%%%%%%%%%%%%
%% I, the copyright holder of this work, release this work into the
%% public domain. This applies worldwide. In some countries this may
%% not be legally possible; if so: I grant anyone the right to use
%% this work for any purpose, without any conditions, unless such
%% conditions are required by law.
%%%%%%%%%%%%%%%%%%%%%%%%%%%%%%%%%%%%%%%%%%%%%%%%%%%%%%%%%%%%%%%%%%%%

\documentclass[
  digital, %% This option enables the default options for the
           %% digital version of a document. Replace with `printed`
           %% to enable the default options for the printed version
           %% of a document.
  twoside, %% This option enables double-sided typesetting. Use at
           %% least 120 g/m² paper to prevent show-through. Replace
           %% with `oneside` to use one-sided typesetting; use only
           %% if you don’t have access to a double-sided printer,
           %% or if one-sided typesetting is a formal requirement
           %% at your faculty.
  table,   %% This option causes the coloring of tables. Replace
           %% with `notable` to restore plain LaTeX tables.
  lof,     %% This option prints the List of Figures. Replace with
           %% `nolof` to hide the List of Figures.
  lot,     %% This option prints the List of Tables. Replace with
           %% `nolot` to hide the List of Tables.
  %% More options are listed in the user guide at
  %% <http://mirrors.ctan.org/macros/latex/contrib/fithesis/guide/mu/fi.pdf>.
]{fithesis3}
%% The following section sets up the locales used in the thesis.
\usepackage[resetfonts]{cmap} %% We need to load the T2A font encoding
\usepackage[T1,T2A]{fontenc}  %% to use the Cyrillic fonts with Russian texts.
\usepackage[
  main=english, %% By using `czech` or `slovak` as the main locale
                %% instead of `english`, you can typeset the thesis
                %% in either Czech or Slovak, respectively.
]{babel}        %% foreign texts to be typeset as follows:
%%
%%   \begin{otherlanguage}{german}  ... \end{otherlanguage}
%%   \begin{otherlanguage}{russian} ... \end{otherlanguage}
%%   \begin{otherlanguage}{czech}   ... \end{otherlanguage}
%%   \begin{otherlanguage}{slovak}  ... \end{otherlanguage}
%%
%% For non-Latin scripts, it may be necessary to load additional
%% fonts:
\usepackage{paratype}
%%
%% The following section sets up the metadata of the thesis.
\thesissetup{
    date          = \the\year/\the\month/\the\day,
    university    = mu,
    faculty       = fi,
    type          = bc,
    author        = {Tereza Šťastná},
    gender        = f,
    advisor       = {doc. RNDr. Jan Strejček, Ph.D.},
    title         = {Simplification\\ of Acceptance Condition\\ of Emerson-Lei Automata},
    TeXtitle      = {Simplification\\ of Acceptance Condition\\ of Emerson-Lei Automata},
    keywords      = {keyword1, keyword2, ...},
    TeXkeywords   = {keyword1, keyword2, \ldots},
    abstract      = {%
      This is the abstract of my thesis, which can

      span multiple paragraphs.
    },
    thanks        = {%
      These are the acknowledgements for my thesis, which can

      span multiple paragraphs.
    },
    bib           = bibliography.bib,
}
\usepackage{makeidx}      %% The `makeidx` package contains
\makeindex                %% helper commands for index typesetting.
%% These additional packages are used within the document:
\usepackage{paralist} 
\usepackage{amsmath}  
\usepackage{amsthm}
\usepackage{amssymb}
\usepackage{amsfonts}
\usepackage{url}      
\usepackage{markdown} 
\usepackage{tabularx} 
\usepackage{tabu}
\usepackage{booktabs}
\usepackage{listings} 
\usepackage{tikz}
\usetikzlibrary{automata,nicearrows,bending}
\lstset{
  basicstyle      = \ttfamily,
  identifierstyle = \color{black},
  keywordstyle    = \color{blue},
  keywordstyle    = {[2]\color{cyan}},
  keywordstyle    = {[3]\color{olive}},
  stringstyle     = \color{teal},
  commentstyle    = \itshape\color{magenta},
  breaklines      = true,
}
\usepackage{floatrow} %% Putting captions above tables
\floatsetup[table]{capposition=top}
%% The following code fixes the rendering of BibLaTeX ISO 690
%% references in old TeX Live (such as the one at Overleaf).
\thesisload
\makeatletter
\def\thesis@biblatexiso@fix@package{iso-numeric.bbx}
\def\thesis@biblatexiso@fix@end{\relax}
\newif\ifthesis@biblatexiso@fix@
\thesis@biblatexiso@fix@false
\def\thesis@biblatexiso@fix@next#1,{%
  \def\thesis@biblatexiso@fix@current{#1}%
  \ifx\thesis@biblatexiso@fix@current\thesis@biblatexiso@fix@package
    \thesis@biblatexiso@fix@true
  \fi
  \ifx\thesis@biblatexiso@fix@current\thesis@biblatexiso@fix@end
    \expandafter
    \@gobble
  \fi
  \thesis@biblatexiso@fix@next
}
\expandafter\expandafter\expandafter\thesis@biblatexiso@fix@next\@filelist,\relax,
\ifthesis@biblatexiso@fix@
  \defbibenvironment{bibliography}
    {\list%
       {\MethodFormat}%
       {\setlength{\labelwidth}{\labelnumberwidth}%
        \setlength{\leftmargin}{\labelwidth}%
        \setlength{\labelsep}{\biblabelsep}%
        \addtolength{\leftmargin}{\labelsep}%
        \setlength{\itemsep}{\bibitemsep}%
        \setlength{\parsep}{\bibparsep}}%
        \renewcommand*{\makelabel}[1]{\hss##1}
        }%
    {\endlist}%
  {\item}%
\fi
\makeatother
\begin{document}
% Tikz styles 
\tikzset{
  collacc0/.style={fill=blue!50!cyan},
  collacc1/.style={fill=magenta},
  collacc2/.style={fill=orange!90!black},
  collacc3/.style={fill=green!70!black},
  collacc4/.style={fill=blue!50!black},
}

\tikzstyle{accset}=[
  circle,inner sep=.9pt,draw=white,
  collacc0, text=white,
  thin,
  anchor=center,
  font=\bfseries\sffamily\scriptsize,
  minimum size={6pt},
]

\newcommand{\accmarkblue}{\begin{tikzpicture} \node[accset, collacc0] {}; \end{tikzpicture}}
\newcommand{\accmarkmag}{\begin{tikzpicture} \node[accset, collacc1] {}; \end{tikzpicture}}
\newcommand{\accmarkor}{\begin{tikzpicture} \node[accset, collacc2] {}; \end{tikzpicture}}

\chapter{Introduction}
\addcontentsline{toc}{chapter}{Introduction}

Our society has become heavily reliant on computer systems in order to function. Unexpected behavior of software often results merely in minor inconvenience. However is some cases, it might pose a threat to human safety. Therefore it is crucial to have the ability to verify if the properties of software in question behave as intended. 

One such verification method is model-checking. The basic idea of this method is using finite state automata to accurately describe the current state of the given program at any time during its execution. The states of the automaton represent the current state of the program, and the transitions between them describe how specific events shift the state of the program from one to another. To be able to describe programs that should never terminate, we use $\omega$-automata -- finite automata on infinite words. During the process of model-checking, a product of two automata is created. It is desirable for the product to be as small as possible, which can be achieved by applying reduction methods to the original automata. 

For some reduction techniques, such as simulation-based reduction and degeneralization, it is beneficial to minimize the number of acceptance sets in the automaton. In \cite{spin2013}, the authors introduce the SCC-based simplifications of acceptance conditions. It is a method that proposes how to reduce the number of acceptance sets in Transition-based Büchi automata (TGBA). For each strongly connected component, it evaluates the relation between acceptance sets and removes the useless ones. 

The goal of this thesis is to propose and implement an analogous simplification method for transition-based Emerson-Lei automata (TELA). This method includes techniques to identify useless acceptance sets and modifications to the acceptance condition formula. During this process, we obtain separate acceptance condition formulas for each accepting SCC. Therefore, the presented method also describes how to merge these formulas into one. Finally, it describes the necessary modifications which ensure that the simplified automaton and the original automaton are equivalent. The desired outcome is to reduce the number of unique acceptance sets in the automaton. 

This thesis is structured into several chapters. Chapter two contains the basic definitions used throughout the thesis. Chapter three summarizes the ideas of the SCC-based simplification method from \cite{spin2013}.  Chapter four presents the SCC-based simplification of TELA. Chapter five discusses the implementation specifics of this method. Chapter six presents the results of experimental evaluation of the implemented tool. Chapter seven contains the conclusion. TODO: add more details

\chapter{Basic definitions}
\label{chap:basic_definitions}
TODO: popis obsahu kapitoly

\section{Transition-based Emerson-Lei Automaton (TELA)} 
\label{sec:tela}
A transition-based Emerson-Lei automaton, as defined in \cite{bloemen2017}, is a tuple $\mathcal{A} = (\Sigma, Q, q_0, \delta, Acc)$, where
\begin{itemize}
  \item $\Sigma$ is an alphabet,
  \item $Q$ is a finite set of states,
  \item $q_0 \in Q$ is the initial state,
  \item $\delta \subseteq Q \times \Sigma \times Q$ is a transition relation,
  \item $Acc$ is the acceptance condition formula, defined as a positive Boolean function over terms of the form Fin$(T)$ or Inf$(T)$ for any subset $T \subseteq \delta$. %positive Boolean function 
\end{itemize}

For a transition $t = (t^s, t^l, t^d) \in \delta$, $t^s$ denotes its source, $t^l$ its label and $t^d$ its destination.

$\rho \in \delta^\omega$ where $\rho(0)^s = q_0 \wedge \forall i \geq 0 : \rho(i)^d = \rho(i + 1)^s$ is a run of $\mathcal{A}$ over $\Sigma^\omega$.

The acceptance of a run $\rho$ is defined by evaluating $Acc$ over $\rho$ where
\begin{itemize}
  \item Fin$(T)$ is true if all the transitions in $T$ occur finitely often, 
  \item Inf$(T)$ is true if some transition in $T$ occurs infinitely often.
\end{itemize}

In figures, an acceptance set is represented by acceptance marks (for example \accmarkblue) placed on the transitions which belong to that set.

\section{Transition-based Generalized Büchi Automaton (TGBA)}
\label{sec:tgba}
Transition-based Generalized Büchi Automaton, as defined in \cite{bloemen2017}, is a TELA $\mathcal{A'} = (\Sigma, Q, q_0, \delta, Acc)$ where 
\begin{equation*}
  Acc = \text{Inf}(T_1) \wedge \text{Inf}(T_2) \wedge \dots \wedge \text{Inf}(T_n)
\end{equation*} 
for some $n$. A run of TGBA over an infinite word is accepting if some transition in $T(i)$ occurs infinitely often for all $i \in \{1, 2, \dots, n\}$. 

\chapter{SCC-based simplification of acceptance conditions of TGBA} 
\label{chap:tgba_simpl}
Let $\mathcal{A'}$ be a TGBA with $n$ acceptance sets: 
\begin{equation*}
  Acc = \text{Inf}(T_1) \wedge \text{Inf}(T_2) \wedge \dots \wedge \text{Inf}(T_n).
\end{equation*}
Let $m$ be the number or accepting SCCs in $\mathcal{A'}$ and $A_\delta = \{A_1, \dots, A_m\}$ be the set of all transitions induced by the accepting SCCs of $\mathcal{A'}$. 
Any accepting run of $\mathcal{A'}$ will be contained in some accepting SCC. Therefore any transitions that are not in $A_\delta$ can be removed without changing the language accepted by $A'$. We can modify the acceptance sets of $\mathcal{A'}$ as follows: 
\begin{equation*}
  Acc = \text{Inf}(T_1 \cap A_\delta) \wedge \text{Inf}(T_2 \cap A_\delta) \wedge \dots \wedge \text{Inf}(T_n \cap A_\delta).
\end{equation*}
If $Acc$ contains such sets that $T_i \subseteq T_j$ and $i \neq j$, we can remove $T_j$, because any run that visits $T_i$ infinitely often also visits $T_j$ infinitely often. Therefore, removing $T_j$ will not change the language.
For each accepting SCC we can define a set of indices of useless acceptance sets 
\begin{equation*}
  \begin{aligned}
    U_k = &\{ j \in \{1, 2, \dots, n\} \mid \exists i \in \{1, 2, \dots, n\}, \\
    &(T_i \cap A_k \subsetneq T_j \cap A_k) \vee (T_i \cap A_k = T_j \cap A_k \wedge j > i)\}.
  \end{aligned}
\end{equation*}
If the two sets $T_i$ and $T_j$ are equal, then, by this definition, only one of their indices belongs to $U_k$, ensuring that one of the sets will be preserved. The set of indices of needed acceptance sets for any accepting SCC is $N_k = \{1, 2, \dots, n\} \smallsetminus U_k$. 

This may result in each SCC having a different number of acceptance sets, however the automaton as a whole can only have one set of acceptance sets. The number of acceptance sets needed in the simplified automaton is $n' = max_{k \in \{1, 2, \dots, n\}} |N_k|$. Let $N'_k$ be a copy of $N_k$. By adding $n' - |N_k|$ indices from $U_k$ to $N'_k$, we ensure that $|N'_k| = n'$ for each accepting SCC. Let $\alpha_k \colon \{1, \dots, n'\} \to N'_k$ be any bijection. Now we can define the new acceptance sets of the simplified automaton $Acc = \text{Inf}(T'_1) \wedge \text{Inf}(T'_2) \wedge \dots \wedge \text{Inf}(T'_{n'})$ as:
\begin{equation*}
  T'_i = \underset{k \in \{1, \dots, m\}}\bigcup (T_{\alpha_k(i)} \cap A_k)
\end{equation*}

\begin{figure}[h]
  \begin{center}
    \begin{tikzpicture}[->,>=stealth,shorten >=1pt,very thick,initial text={}]
      \tikzstyle{every state}=[very thick]
      \node[state, initial above]   (0) at (0,0) {0};
      \node[state]            (1) at (-3, 0) {1};
      \node[state]            (2) at (3, 0) {2};

      \path[->, very thick]
              (0) edge[bend right]  node [above] {\texttt{tt}} (1) 
                  edge[bend left]   node [above] {\texttt{tt}} (2) 
              (1) edge[loop above]  node[above] {$ab$} node[accset, collacc1, pos=0.40] {} node[accset, collacc0, pos=0.15] {} node[accset, collacc2, pos=0.75] {} (1) 
                  edge[loop left]   node[left] {$a\overline{b}$} node[accset, collacc1] {} node[accset, collacc2, pos=0.15] {} (1) 
                  edge[loop right]  node[right] {$\overline{a}\overline{b}$} node[accset, collacc1] {} (1) 
                  edge[loop below]  node[below] {$\overline{a}b$} node[accset, collacc1] {} node[accset, collacc0, pos=0.15] {} (1) 
              (2) edge[loop above]  node[above] {$c$} node[accset, collacc1, pos=0.40] {} node[accset, collacc0, pos=0.15] {} node[accset, collacc2, pos=0.75] {} (2) 
                  edge[loop below]  node[below] {$\overline{c}$} node[accset, collacc0] {} node[accset, collacc2, pos=0.15] {} (2);    
    \end{tikzpicture}
    \caption{TGBA with three acceptance sets}
    \label{fig:tgba}
  \end{center}
\end{figure}

In Figure \ref{fig:tgba} there is a TGBA with two accepting SCCs. Let $S_1$ be the SCC comprised of state 1 and $S_2$ be the SCC comprised of state 2. The set of useless acceptance sets for $S_1$ is $U_1 = \{\accmarkmag\}$, because $\accmarkblue \subseteq \accmarkmag$ (and also $\accmarkor \subseteq \accmarkmag$). For $S_2$ it is $U_2 = \{\accmarkblue, \accmarkor\}$, because $\accmarkmag \subseteq \accmarkblue$ and $\accmarkmag \subseteq \accmarkor$.

The sets of needed acceptance sets for $S_1, S_2$ are $N_1 = \{\accmarkblue, \accmarkor\}$, $N_2 = \{\accmarkmag\}$ respectively. The number of needed acceptance sets for the whole automaton is $n' = 2$. We can define $N'_1 = N_1, N'_2 = N_2 \cup \{\accmarkor\}, \alpha_1 (\accmarkor) = \accmarkor, \alpha_1 (\accmarkblue) = \accmarkblue, \alpha_2 (\accmarkor) = \accmarkor, \alpha_2 (\accmarkblue) = \accmarkmag$. The result is the TGBA in Figure \ref{fig:simpl_tgba}.

\begin{figure}[h]
    \begin{center}
    \begin{tikzpicture}[->,>=stealth,shorten >=1pt,very thick,initial text={}]
      \tikzstyle{every state}=[very thick]
      \node[state, initial above]   (0) at (0,0) {0};
      \node[state]            (1) at (-3, 0) {1};
      \node[state]            (2) at (3, 0) {2};

      \path[->, very thick]
              (0) edge[bend right]  node [above] {\texttt{tt}} (1) 
                  edge[bend left]   node [above] {\texttt{tt}} (2) 
              (1) edge[loop above]  node[above] {$ab$} node[accset, collacc0, pos=0.15] {} node[accset, collacc2] {} (1) 
                  edge[loop left]   node[left] {$a\overline{b}$} node[accset, collacc2] {} (1) 
                  edge[loop right]  node[right] {$\overline{a}\overline{b}$} (1) 
                  edge[loop below]  node[below] {$\overline{a}b$} node[accset, collacc0] {} (1) 
              (2) edge[loop above]  node[above] {$c$} node[accset, collacc0, pos=0.15] {} node[accset, collacc2] {} (2) 
                  edge[loop below]  node[below] {$\overline{c}$} node[accset, collacc2] {} (2);    
    \end{tikzpicture}
    \caption{TGBA from Figure \ref{fig:tgba} after SCC-based acceptance simplification}
    \label{fig:simpl_tgba}
  \end{center}
\end{figure}


\chapter{SCC-based simplification of acceptance conditions of TELA}
\label{chap:simpl_tela}
TODO: chapter content

In this chapter, let the input TELA have at least one acceptance set. Let the acceptance condition formula of the TELA be in disjunctive normal form (DNF). 

Because the formula is in DNF, we can represent it as $Acc = \{d_1, d_2, \dots, d_m\}$ , where each $d_i$ is a disjunct. Further, each such disjunct $d_i$ is represented as a set of conjuncts. For example, formula $Acc_e = (\text{Inf}(T_1) \wedge \text{Fin}(T_2)) \vee (\text{Inf}(T_3))$ is represented by $Acc_e = \{d_1, d_2\}$, where $d_1 = \{\text{Inf}(T_1), \text{Fin}(T_2)\}, d_2 = \{\text{Inf}(T_3)\}$. This type of notation is used throughout the chapter.

\section{Simplification of SCCs}
Unlike TGBA, TELA have complex acceptance condition formulae. Therefore, to identify which acceptance sets are useless, we need to consider not only the relation between the transitions in these sets but also the specifics of their occurrence in the acceptance condition formula. 

For example, consider acceptance sets $T_1, T_2, T_3$ and $T_1 \subseteq T_2$ and $Acc_a = \text{Inf}(T_1) \wedge \text{Inf}(T_2)$. In this case, we can apply the simplifications from Chapter \ref{chap:tgba_simpl} and remove $T_2$, because the form of $Acc_a$ is identical with the form of TGBA formula. However, in $Acc_b = \text{Inf}(T_1) \vee (\text{Inf}(T_2) \wedge \text{Fin}(T_3))$ simply removing $T_2$ could change the language. This example demonstrates that for SCC-based TELA simplifications, it is necessary to carefully define the requirements about the form of the acceptance condition formula.

\subsection{Inclusion-based simplifications}
Let $T_1 \subseteq T_2$ be two distinct acceptance sets that appear in the acceptance condition formula only in terms Inf$(T_1)$ and Inf$(T_2)$. We can modify the simplifications from Chapter \ref{fig:simpl_tgba} to make them viable for TELA by adding a prerequisite that demands a specific form of the acceptance formula. The requirement is that for any disjunct in the formula, either both sets are present, or neither is present. If the required conditions are met, we may remove $T_2$ from the formula. 

More formally, if the following conditions hold
\begin{itemize}
  \item $i, j \in \{1, 2, \dots, n\}$,
  \item $(T_i \subsetneq T_j \vee (T_i = T_j \wedge j > i))$,
  \item $\forall d_k \in Acc \colon (\text{Inf}(T_i) \in d_k \Leftrightarrow \text{Inf}(T_j) \in d_k)$,
\end{itemize}
then we can remove Inf$(T_j)$ from $Acc$: 
\begin{equation*}
  Acc = \{d_1 \smallsetminus \{\text{Inf}(T_j)\}, d_2 \smallsetminus \{\text{Inf}(T_j)\}, \dots, d_m \smallsetminus \{\text{Inf}(T_j)\}\}.
\end{equation*} 


\subsection{Simplifications of complementary acceptance sets}

\subsection{Acceptance condition formula modifications}
The following modification of the acceptance condition formula is not affected by any specific relation between acceptance sets. It can be applied if there are such disjuncts $d_k, d_l$ in the $Acc$ that both contain the acceptance set $T_i$, and $d_k$ only contains one item, i.e., it only contains $T_i$. In this case, it does not matter if the acceptance set $T_i$ appears in the $Acc$ as Inf$(T_i)$ or Fin$(T_i)$. The cardinality of $d_l$ is arbitrary. Under these conditions, $d_l$ can be removed from the $Acc$. This modification does not change the -- value of the $Acc$ formula, because either $d_k$ is satisfied and the whole formula is evaluated as true regardless of $d_l$, or $d_k$ is not satisfied, in which case $d_l$ is also not satisfied. 
Formally, if these conditions are met:
\begin{itemize}
  \item $i \in \{1,2, \dots, n\}$,
  \item $k,l \in \{1,2, \dots, m\}$,
  \item TODO: define C,
  \item $\exists d_k, d_l \in Acc \colon |d_k| = 1 \wedge (\exists C \colon C = \text{Tnf}(T_i) \wedge \in d_k \wedge C \in d_l)$,
\end{itemize}
we may remove $d_l$ from $Acc$:
\begin{equation*}
  Acc = Acc \ \{d_j\}
\end{equation*}




\section{Merging acceptance condition formulae of all SCCs}

\section{Restoring equivalence with the original automaton}


\chapter{Implementation}


\chapter{Experimental Evaluation}


\chapter{Conclusion}



\printbibliography[heading=bibintoc]

\end{document}
